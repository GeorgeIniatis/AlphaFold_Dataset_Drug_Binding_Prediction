\section{Introduction}
\label{sec:Introduction}

Drug-target interactions (DTIs) refer to the interactions of chemical compounds and biological targets, proteins in our case, inside the human body \citep{Sachdev2019}. Given that both proteins and drugs are chemically active molecules in the bloodstream, it would make sense that they interact in some way \citep{DerangedPhysiology}. These interactions usually form an ever-changing, benign and reversible binding where both molecules move through the bloodstream interlocked together \citep{DerangedPhysiology}. The vast majority of drugs administered take this into account and use this process \citep{DerangedPhysiology}. A protein-bound drug is usually too big to pass through a biological membrane like that of a cell. Therefore only the unbound drug, usually in equilibrium with the bound drug, can pass through and produce the desired pharmacological effect, like the treatment of a disease, or the targeting of a tumour \citep{ProteinBindingOverview}. This is done by binding and usually inactivating another protein by inhibiting its function \citep{Lu2020}. This inhibition depends on the specific drug-protein pair and it can take the form of actively blocking the protein's binding sites, altering the protein's structure or preventing the protein from transmitting chemical signals, amongst many others \citep{Mozhaev1982}. Examples include antibiotics and protease inhibitors that have been widely used to combat diseases like Covid-19, Hepatitis C and HIV \citep{Berry2022, Ma2022}.

\hspace{1cm}

Consequently, the degree of how much a drug binds to a protein can enhance or diminish the drug's effectiveness and performance. For example, minimally protein-bound drugs tend to penetrate tissue better and are excreted much faster from the body than those highly bound \citep{Scheife1989}. In contrast, highly protein-bound drugs, usually meaning that the protein binding is so impactful that we have to pay attention to it, tend to last much longer. This is because the protein acts as a drug "depot" that slowly releases the drug into the bloodstream, again keeping the bound and free drug in equilibrium \citep{DerangedPhysiology, ProteinBindingOverview}.

DTIs play a crucial role in drug discovery and pharmacology. However, the experimental determination of these interactions with methods, such as fluorescence assays, is time-consuming and limited due to funding and the difficulty of purifying proteins \citep{Shar2016, Wang2020}. Past quantitative structure-relationship activity (QSAR) studies discussing protein-drug binding focused on testing thousands of drugs with just a single protein that they deemed important enough \citep{Colmenarejo2003, Vallianatou2013}. These studies often did not consider the protein's sequence or structural information, concentrating their efforts on the drug molecules and their descriptors. However, this is not what we aimed to do in this study. Unwanted or unexpected DTIs could cause severe side effects, therefore, the creation of in silico machine learning models with high throughput that can quickly and confidently predict whether thousands of drugs and proteins bind together and how much could be crucial for medicinal chemistry and drug development, acting as a supplement to biological experiments \citep{Shar2016, Wang2020}.

\subsection{Objectives}
\label{subsec:Objectives}

The project aims to gather publicly available data on known drug-target interactions and place them into a new curated dataset. Then, using this new dataset, train multiple machine learning models using QSAR descriptors derived from a drug's chemical properties and a protein's sequence and 3D structural information to predict whether they bind together. Each protein's 3D structure will be extracted from the AlphaFold protein structure database \citep{Jumper2021, Varadi2022} and one of the main challenges of the project will be in creating an embedding, which efficiently encodes the structural information of the protein, that can then be used in our training process. The models' performance should then be evaluated, and a rudimentary system using these models should be constructed.

\hspace{1cm}
